%
% c't-Bot
% 
% This program is free software; you can redistribute it
% and/or modify it under the terms of the GNU General
% Public License as published by the Free Software
% Foundation; either version 2 of the License, or (at your
% option) any later version. 
% This program is distributed in the hope that it will be 
% useful, but WITHOUT ANY WARRANTY; without even the implied
% warranty of MERCHANTABILITY or FITNESS FOR A PARTICULAR 
% PURPOSE. See the GNU General Public License for more details.
% You should have received a copy of the GNU General Public 
% License along with this program; if not, write to the Free 
% Software Foundation, Inc., 59 Temple Place, Suite 330, Boston,
% MA 02111-1307, USA.
%

\documentclass{article}
\usepackage[a4paper,top=2cm,bottom=2cm,left=1cm,right=1cm]{geometry}
\usepackage[utf8x]{inputenc}
\usepackage{graphicx}
\usepackage[german]{babel}
\usepackage[T1]{fontenc}

\begin{document}
\thispagestyle{empty}

\begin{figure}
 \begin{center}
   \begin{tabular}{cc}
     \resizebox{180mm}{!}{\includegraphics[angle=0,scale=1]{plots.pdf}}
   \end{tabular}
   \caption{Soll-Ist-Geschwindigkeit zu PWM Wert}
   \label{Diagramme}
 \end{center}
\end{figure}


\section*{Diagramm Auswertung der Motorregelung}


Bei den oben gezeigten Diagrammen handelt es sich um den zeitlichen Verlauf der Motorleistung in [PWM] bei entsprechender Soll-Ist Geschwindigkeit in [mm/s] f"ur den linken und rechten Motor. Die Messwerte wurden mittels Odometrie und Speedlog vom Roboter selbst aufgenommen und nach Messfahrt am PC ausgewertet.

\paragraph{Datum} \today
\paragraph{Messung} Messfahrt mit neuen PID-Parametern
\paragraph{Bewertung} OK


\end{document}